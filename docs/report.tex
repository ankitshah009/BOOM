\documentclass{article}
\usepackage{hyperref}
\usepackage{amsmath}
\usepackage{amssymb}
\usepackage{color}
\usepackage{tabularx}
\usepackage{graphicx}
\usepackage{enumerate}
\usepackage{listings} 
\usepackage{float}

 
\DeclareMathOperator{\E}{\mathbb{E}}
\DeclareMathOperator{\Prob}{\mathbb{P}}

\begin{document}

\title{11-792 Project Report}
 
\author{Nicholas Gekakis, Boyue Li}
 
\maketitle
 
\section{Overview}

In this project, we are building a distributed question answering pipeline framework,
which allows users to easily configure multiple modules,
create complex pipelines and tune parameters automatically.

\section{Requirements}

    \subsection{Easy to configure and deploy}
    The framework should be easy to configure and deploy.

    \subsection{Save and resume}
    The framework should be able to save intermediate results to resume training.

    \subsection{Pipeline topology}
    The framework should be able to create pipelines with complex topologies.

    \subsection{Automatically parameters tuning}
    The framework should be able to automatically tune some parameters.

    \subsection{Distributed parallel processing}
    The framework should support distributed parallel processing to handle large datasets and complicated pipelines.

\section{Design}
    A pipeline is constructed by several independent modules,
    users only need to specify the correspondance between inputs and outputs,
    the framework would automatically handle all calculation.

    \subsection{Module}
    A module is the basic calculation unit which takes arbitrary number of inputs and produces arbitrary number of outputs.
    Every input and output is an information object defined below.
    A module also needs a description file to specify following fileds:
    \begin{itemize}
        \item Number of inputs.
        \item Number of outputs.
        \item Data type of inputs.
        \item Data type of outputs.
        \item Number of parameters.
        \item Default values of parameters.
        \item Tuning interval of parameters.
        \item Tuning steps of parameters.
    \end{itemize}

    \subsection{Information object}
    The information obejct used to pass data between modules contains following fileds:

    \begin{itemize}
        \item Producing module: the module that produced this information object.
        \item Consuming module: the module that this information object to be passed to.
        \item Data path: the path to actual data file.
        \item Data type: the type of data (one of number, binary and string or user defined data type).
        \item Data size: the number of data instances.
    \end{itemize}

    \subsection{File format}
    A data file has a description file which contains following fileds:

    \begin{itemize}
        \item The configuration applied to it.
        \item The timestamp when it is created.
        \item Data type.
        \item Data size.
    \end{itemize}

    \subsection{Configuration file}
    We use YAML files to configure the framework.

    \subsection{Code structure}


\section{Experiments}
    \subsection{Dataset1}
    \subsection{Dataset2}
    \subsection{Dataset3}

\section{Conclusion}


\section*{Acknowledgements}

\end{document}



